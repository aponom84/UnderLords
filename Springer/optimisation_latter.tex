%\documentclass{article}
\documentclass[smallextended]{svjour3}       % onecolumn (second format)
%\usepackage{arxiv}
\usepackage[utf8]{inputenc} % allow utf-8 input
%\usepackage[T1]{fontenc}    % use 8-bit T1 fonts
\usepackage{hyperref}       % hyperlinks
\usepackage{url}            % simple URL typesetting
\usepackage{booktabs}       % professional-quality tables
\usepackage{amsfonts}       % blackboard math symbols
\usepackage{nicefrac}       % compact symbols for 1/2, etc.
\usepackage{microtype}      % microtypography
\usepackage{lipsum}

%\usepackage{amsmath, amsfonts, amsthm,amssymb}
\usepackage{amsmath, amsfonts, amssymb}
%\usepackage[english, russian]{babel}
\usepackage[english]{babel}
\usepackage{graphicx}

%\newtheorem{theorem}{Theorem}
%\newtheorem{lemma}{Lemma}
%\newtheorem{corollary}{Corollary}
%\newtheorem{proposition}{Proposition}
%\newtheorem{assertion}{Assertion}

\usepackage{xcolor}

% \title{NP-completeness and LP formulation of Dota Underlords problem}
% \title{Hacking Dota Underlords With Integer Programming Model}
%\title{NP-completeness and linear integer programming model for Dota Underlords problem}
\title{Dota Underlords game is NP-Complete}


\author{Dmitry V. ~Sirotkin \and \\
Alexander A. ~Ponomarenko}


\institute{Dmitry V. ~Sirotkin \at
National Research University Higher School of Economic\\
International Laboratory of Statistical and Computational Genomics\\
34 Tallinskaya ul., Room 707, Moscow, Russia\\
Tel.: +7(495) 772-95-90, ext.15060\\
\email{dsirotkin@hse.ru} \\
\and
Alexander A. Ponomarenko \at
National Research University Higher School of Economics \\
Laboratory of Algorithms and Technologies for Network Analysis\\
136 Rodionova street, Nizhny Novgorod, Russia \\
Tel.:  +7 (831) 436-13-97\\
\email{aponomarenko@hse.ru}\\
\emph{Corresponding Author}
}

\date{Received: date / Accepted: date}


\begin{document}
\maketitle

\begin{abstract}
In this paper, we demonstrate how the optimal team choice problem in the popular computer game Dota Underlords can be reduced to the linear integer programming problem. We propose a model and solve it for the real data. We also prove that this problem is NP-hard and show that it can be polynomially reduced to the maximum edge-weighted clique problem.
\end{abstract}


% keywords can be removed
%\keywords{Целочисленное программирование \and NP-трудность \and NP-полнота}
\keywords{NP-completeness \and Linear Integer Programming \and Computational Complexity \and Video Game \and Optimisation \and Polynomial-Time Reduction \and Maximum Edge-Weighted Clique}

\section{Introduction}
People love to play games. Many games and puzzles that people play are interesting due to their complexity: you need to be smart enough to solve it. In many cases, such complexity can be expressed as computation complexity depending on the input size. For example, it was shown that the puzzle ``Sokoban" is polynomially solvable \cite{hearn2005pspace}; in \cite{fraenkel1981computing} authors proved that the chess game belongs to the EXPTIME complexity class.
In turn, NP-complete computational class has a special place in theoretical computer science.
In \cite{breukelaar2004tetris} was proved that the optimally playing offline version of legendary video game ``Tetris" is NP-complete. As well, it was found in \cite{kaye2000minesweeper} that ``Minesweeper" belongs to NP-complete class. Decision version of the problem of finding a minimal number of chip movements in a generalized version of 15-puzzle for the board with size $N \times N$ belongs is NP-complete also \cite{ratner1986finding}.\\
In some sense, every NP-complete problem is a puzzle, and vice-verse, many puzzles are NP-complete. We refer reader to the surveys \cite{costa2018computational}, \cite{kendall2008survey} for deeper study computational complexity of puzzles and games.\\
In this paper, we consider the popular video game Dota Unlderlords. It is one of the so-called auto-chess games. It turns out that this problem can be represented as a combinatorial optimization problem, which decision version  belongs to the class NP-complete.\\
The article is organized as follows. In section \ref{SectionDUDescription} Dota Underlords gameplay is described.
We present a formulation of the Dota Underlords problem (DU problem) as the linear integer programming model in section \ref{SectionDUIP}. In the \ref{SectionNPCompleteProof} section, we show the NP-completeness of this task and reduce the problem to the maximum edge-weighted clique problem (MEWC). The solution of the integer programming model for a real data is presented in section \ref{SectionComputationalResults}. Traditionally, we summarize in the section \ref{SectionConclusion} ``Conclusion".

\section{Dota Underlords game play description}
\label{SectionDUDescription}
During a game eight players build a team of ``heroes"-- creatures that can fight each other on a game map. Each of the heroes has basic parameters: health, damage, attack speed, and others, as well as a special ability that determines its role in the game. Each hero belongs to two or more ``alliances"-- the sets that unite several heroes. For example, Enchantress hero belongs simultaneously to the ``druids" alliance and to the ``predators" alliance. When there are several heroes in the team who belong to the same alliance (for each alliance this number is individual), the player receives a bonus consisting of improving the characteristics of his heroes or worsening the parameters of enemy's heroes.\\
In a game, a player can strengthen heroes by upgrading them to higher levels or by purchasing in-game items. Moreover, during the game player can change the line-up of the team by spending in-game money in order to react on the actions of his opponents. However in turn it may weaken the team, because somewhat weaker heroes can appear in the game later. Also, during a combat the position of heroes on a board matters. In this work, we don't take into account these aspects. The reason is that it is very hard to model all of the intricacy of the game due to involving many factors. Therefore the goal of our model is to analyse the core gameplay which defines the auto-battler genre -- the forming a team of heroes in order to maximize a total strength.

Thus, in our simplified model of the real game, the strength of a team is determined by two factors:
\begin{enumerate}
    \item The power of selected heroes
    \item Bonuses from the alliances which they are belong
\end{enumerate}


\section{Problem reformulation in terms of linear integer programming}
\label{SectionDUIP}

\subsection{The simplest problem statement}

We formalize the problem as follows. We assume that in total we have $n$ heroes to choose from. We assume that the strength (power) of some $i$-th hero is presented by some non-negative value $s_i$. $x_i$ indicates the belonging of hero $i$ to the team being formed. We suppose $ x_i = 1 $, if the $ i $ -th hero belongs to the team, and $ x_i = 0 $ otherwise.  The condition that there are no more than $m$ heroes in a team, can be written as $ \sum_{i=1}^n x_i \leq m $.
Therefore, the simplest way to formalize the problem is following:

\begin{equation}
\begin{gathered}
    max \sum_{i=1}^n x_i s_i \\
    \sum_{i=1}^n x_i \leq m \\
    x_i \in \{0, 1\} \text{ – decision variable} \\
   n, m, s_i \text{ – constants}  \\
\end{gathered}
\end{equation}

A trivial solution of this problem is to take $ n $ elements with the largest weights.
\subsection{Problem statement with alliances}
As mentioned above, in ``Dota Underlords" game each hero belongs to two or more ``alliances" --- in turn, each alliance includes several heroes. When the team has several heroes from the same alliance (for each alliance this number is individual), the player receives a bonus that either strengthens all the heroes of the alliance, or strengthens all heroes of the team, or weakens all the heroes of the opponent team. The latter can be interpreted as a relative strengthening of own heroes. Therefore we consider only the first two cases throughout the paper. It should be noted that an alliance can have several bonuses that are unlocked by the different numbers of heroes of that alliance. These bonuses can also be of various types.\\
We propose to model these cases by introducing 3-index tensor $ e_{ijk} \in \mathbb{R} $ which represents a bonus to hero $i$ from alliance $j$, in which there are at least $k$ heroes of the alliance $j$. In other words, $ e_{ijk} $ is the $k$-th bonus of alliance $ j$ for hero $i$.\\
By using tensor $e_{ijk} $, we support both types of bonues: the first gives bonuses to its members, and the second provides bonuses to all heroes of the team. Moreover, alliances of both types differ in only one thing. The value of $e_{ijk} $ is zero for alliances that provide bonus to their members, if and only if the $i$-th hero doesn't belong to the $j$-th alliance. In the general case, it is not necessarily true. Note that the tensor $ e_{ijk} $ is sparse for the real instances of auto-chess games genre since there are few alliances that give bonuses to all the heroes in the team.
\\ We use control binary variable $ I_{ijk} $ to indicate if the bonus $ e_{ijk} $ is active.
Therefore we can write down the objective function as the following sum $ \sum_{i = 1}^{n} x_i s_i + \sum_{i=1}^{n} \sum_{j=1}^{t}  \sum_{k=1}^{q} e_{ijk} I_{ijk} $.
Inequalities $\forall{i,j,k} :  \sum_{i'=1}^{n} a_{i'j} x_{i'} -  k\ge M( I_{ijk} - 1)$ connect the variables $ x_{i} $ and $ I_{ijk} $
\\These inequalities do not allow the binary variable $ I_{ijk} $ to have the value 1 if the solution includes less than $k$ heroes from the alliance $j$. When the solution contains less than $m$ heroes from the alliance $j$, the left side of this inequality is negative. So for the considered inequalities, the right side should be even smaller. It is possible only when the binary variable $ I_{ijk} $ is zero. In this case the right-hand side is $-M $, where $M$ is a big constant known to be larger than $k$ i.e. larger than the maximum size of the alliance $q$.\\
We require that the bonus for the hero $i$ can be activated ($ I_{ijk} = 1 $) only if the hero $i$ belongs to the solution. This is given by the inequalities $ \forall {i, j, k}: I_{ijk} \le x_i $. We also want that the bonus $ e_{ijk} $ to be activated only if the character $ i $ belongs to the alliance $j$. For this, we include in the model inequalities $ \forall{i, j, k} :\, I_{ijk} \le a_{ij} $.\\
Thus, after introducing alliances for the model, the system of equations can be written as the following:
\begin{equation}
\label{eq:DUIP}
\begin{gathered}
\textbf{Objective function}\\
max \sum_{i=1}^{n} x_i s_i + \sum_{i=1}^{n} \sum_{j=1}^{t}  \sum_{k=1}^{q} e_{ijk} I_{ijk} \\
\textbf{Constraints for the input data}\\
\forall{j} : \sum_{i=1}^n a_{ij} \le q \\
\textbf{Constraints for the decision variables} \\
\forall{i,j,k} :  \sum_{i'=1}^{n} a_{i'j} x_{i'} - k \ge M( I_{ijk}  - 1) \\
\sum_{i=1}^n x_i \le m   \\
% \sum_{i=0}^{n-1} a_{ij} x_{i} < k \\
\forall{i,j,k} :  I_{ijk}  \le x_i \\
%\forall{i,j,k} :  I_{ijk}  \le a_{ij} \\
\textbf{Decision variables} \\
I_{ijk} \in \{0, 1\} \text {, 1 – if for the hero } i \text{, the } k\text{-th bonus is activated for }  j \text{-th alliance,} \\
x_i  \in \{0, 1\} \text{, 1 -- if hero } i \text{ belongs to solution} \\
\textbf{Constants} \\
n \in \mathbb{N} \text{ -- number of heroues,} \\
m \in \mathbb{N} \text{ -- maximum size of the team}\\
t \in \mathbb{N} \text{ -- the total number of alliances} \\
q \in \mathbb{N} \text{ -- maximum size of an alliance,} \\
s_i  \in \mathbb{R} \text{ –- the strength of the hero } i, \\
e_{ijk} \in \mathbb{R} \text{ -- the bonus for the hero } i \text{,  if } k
\text{-th bonus is activated for the } j \text{-th alliance} \\
a_{ij} \in \{0, 1\} \text{ -- indicates if hero } i \text{ belongs to the alliance } j \\
\end{gathered}
\end{equation}
\section{Proof of an NP-completeness of the DU problem}
\label{SectionNPCompleteProof}
To prove that a problem is NP-complete, it is necessary to show that it is NP-hard  and at the same time it belongs to the NP class. Let's us prove both statements.
\subsection{Polynomial-time reduction from the Maximum Density Sub-Graph Problem to the DU problem}
First of all, let us remind to reader that the problem of finding a set of $k$ vertices with the most edges in the subgraph induced by this set is called the maximum dense subgraph of $k$ vertices problem \cite{kortsarz1993choosing}. Obviously this problem is NP-hard by reduction from clique.
\begin{theorem}
\label{MEWC_DU}
The problem of finding the maximum dense subgraph of $k$ vertices reduces to the DU problem.
\end{theorem}
\begin{proof}
Let consider a special case of the DU problem with the following restrictions:
\begin{enumerate}
    \item The powers of all heroes are the same ($\forall i, j \; s_i=s_j$)
    \item Alliance can provide bonuses only to the heroes of that alliance. ($\forall i, j, k \; a_{ij}=0 \Longrightarrow e_{ijk} = 0$)
    \item All alliances have the size 2 ($\forall j \; \sum_i a_{ij}=2 $)
    \item All alliances give bonus if and only if both heroes present in the team ($\forall i, j \; e_{ij1}=0$)
    \item Bonuses from all alliances are the same ($\forall i, j, i', j' \; a_{ij}=1,\, a_{i' j'}=1 \Longrightarrow e_{ij2}=e_{i' j' 2}$)
\end{enumerate}
Then the data can be represented as a graph $ G(V, E) $, where the set of vertices $ V $ corresponds to the heroes, and the set of edges $E$ corresponds to the activated alliances. You may notice that in this case the optimal team of size $ k $ is the densest subgraph $ G' \subset G $ with $ k $ vertices.  According to this formulation, density can be understood as a value $ \frac{|G'(E)|}{|G'(V)|} $. Indeed, under these restrictions, the total strength of the team linearly depends on the number of activated alliances, which corresponds to $ G'(E)$. Since $ k $ is a constant, the total strength of the team grows with the increasing density of the graph $ G '$ .\\
Thus, we have shown that finding the maximum dense subgraph of $k$ vertices is a special case of the DU problem, so it is reducible to DU problem. Therefore the DU problem is no less difficult than the NP-hard  problem. Consequently the DU problem is NP-hard as well$\blacksquare$
\end{proof}

\subsection{Dota Underlords belongs to NP-class}
Decision version of DU problem (problem with answer ``yes" or ``no") can be formulated as follows: ``\textit{Is there a team with at most $ m $ heroes and with a total power greater than some given constant?}". Then, we are ready to state the following theorem.
\begin{theorem}
\label{DU_is_NP}
The decision version of DU problem belongs to the  NP class.
\end{theorem}
\begin{proof}
By definition of NP-class, the problem belongs to the NP class, when a reported solution can be checked in polynomial time. In our case, solution is a set of $m$ heroes. To verify solution, we need calculate the total power of the team.\\
So, we just need to calculate the objective function. In turn, to do that, at first we need to find out what the alliances are formed. This means that we need to calculate the number of non-zero elements for each column in the matrix $a_{ij}$, taking into account only rows corresponding to heroes from the team i.e  we mean submatrix $\{a_{ij}:  j \in \overline{1,m},\; i \in \{  i' \in \overline{1,n} :   x_{i'} = 1 \}  \}$. It can be done for $O(nt)$ operations. \\
After that, the objective function can be calculated in a straightforward way by $O(ntq)$ number of operations. Thus, we can check the solution for polynomial time dependent on the input size  $\blacksquare$
\end{proof}

\subsection{NP-completeness of DU problem}
\begin{theorem}
	The DU problem defined by the system of inequalities \eqref{eq:DUIP}  belongs to the class of NP-complete problems.
\end{theorem}

\begin{proof}
Theorem \ref{MEWC_DU} states that there exists a polynomial-time reduction of an NP-hard problem to DU. At the same time, in the theorem \ref{DU_is_NP} we showed that the DU problem belongs to the NP class. Thus, the DU problem is NP-hard, and at the same time, it lies in the NP class. Therefore the decision version of DU problem is NP-complete $\blacksquare$
\end{proof}

\section{Polynomial-Time Reduction from the DU problem to the MEWC}
While working on the paper, additionally we found a polynomial-time reduction of DU problem to well-known problem --- Maximum Edge-Weighted Clique (MEWC). Thus anyone can use the already developed algorithms for the MEWC problem such as \cite{san2019new} or an efficient quadratic formulation from \cite{hosseinian2017maximum} for solving instances of the DU problem.\\
In this reduction, we will consider a problem with the maximum size of the alliance bounded by some constant $q$.

\begin{figure}[h!]
\begin{center}
\includegraphics[height=4in,width=3in,angle=0]{figure1.pdf}
\caption{An example of graph $G'$ from the theorem \ref{general_case}  for the case $n=4$, $m=3$, $q=2$.  Graph $G'$ consists of the sets $\mathcal{F}$ and $\mathcal{F}_2$. The edges between every vertex of sets $\mathcal{F}$ and $\mathcal{F}_2$ are avoided for picture simplicity. The size of the maximal clique is 6. }
%в подписи к рисунку надо ещё вставить упоминание о графе G
\label{fig:reduction}
\end{center}
\end{figure}

\begin{theorem}
\label{general_case}
    DU problem with alliances of size $q$, has a polynomial reduction to the MEWC.
\end{theorem}
%	The proof will be constructed similarly to the proof of the theorem \ref {second_case}.
%We will create additional vertices associated with all possible combinations of $q$ heroes. The bonuses from the formation of the corresponding alliances will be the same as in the theorem \ref {second_case} is on the edges.

\begin{proof}

	At first, as a base we build a graph $G$ with weighted edges.
%We construct a graph $G$  such that the solution of the problem DU (Dota Underlords) follows from the solution of the problem MEWC. Moreover, the size of the MEWC problem is limited by a polynomial on the size of the DU problem.
    We construct a set $V^1 $ of $n$ vertices. Each vertex we assign with one of the heroes of the DU problem. In other words, we name each vertex in honour of one of the heroes of the DU problem. We enumerate these vertices according to order of heroes $v_1^1$, $v_2^1$, ..., $ v_n^1$.
    Additionally we construct $m-1$ sets of vertices $ V^2 $, $ V^3 $, and so on up to $ V^m $, In each set we also name every vertex in honour of a hero of  DU problem. Similarly to the first set, we enumerate the vertices in the set $V^i $ as $ v_1^i $, $v_2^i$, ..., $v_n^i$.
    We denote a family of these sets as $ \mathcal{F} $. Therefore we get $m$ sets of $n$ vertices.
   We build edges in the graph $G$ as follows --  an edge is drawn between the vertices $v_a^i $ and $v_{a'}^{i'} $  if both of the following conditions are true:
    \begin{itemize}
        \item The vertices $v_a^i $ and $v_{a'}^{i'} $ correspond to distinct heroes ($a \neq a'$)
        \item The vertices $v_a^i $ and $v_{a'}^{i'} $ belong to different sets of the family $\mathcal{F}$ ($ i \neq i /'$)
    \end{itemize}

    Let consider all maximum cliques in graph $G$. Obviously, in any clique of size $m$, there is exactly one vertex from each set $ V^i $. At the same time all vertexes of a clique correspond to distinct heroes. Therefore each clique corresponds to a team of $m$ heroes of the DU problem. Note that each team can correspond to several cliques.\\
    To introduce heroes power, we assign a weight $ \frac{s_a}{m-1} + \frac{s_{a '}}{m-1}$ to an edge between the vertices $v_a^i$ and $v_{a'}^{i'}$.\\
    Let us show that the sum of the edges weights of a clique is exactly the total power of the team. Indeed, for each vertex of the clique there are exactly $m-1$ incident edges. Then, each term $\frac{s_i}{m-1}$ corresponding to a vertex with a subscript $i$ is included in the total sum exactly $m-1$ times. It follows that the sum of all the weights of the clique edges is the sum of all the values $s_i$ corresponding to the vertices of this clique.\\

Secondly, we construct a graph $G'$ with weighted edges in such a way that the solution of the DU problem follows from the solution of the MEWC problem for graph $G'$. Moreover, the size of the MEWC problem is bounded by a polynomial in the size of the DU problem.\\
    We construct a set $W^{1,2,...,q} $ with $\binom{n}{q} $ vertices, where each vertex corresponds to an unordered set of $q$ heroes. We enumerate the vertices by lexicographic order corresponding to the order of combinations of $ \binom {n}{q} $ elements of $ w^1_ {\underbrace {1,2, ..., q}_\text{total $ q $ indices }}, w^1_{\underbrace {1,2, ..., q-1, q + 1}_\text {total $ q $ indices}}, ..., w^1_ {\underbrace{n-q+1, ..., n-1, n}_\text{total $ q $ indices}} $. It is important that each of these elements has exactly $q$ indices.\\
We additionally construct $ \binom{m}{q}-1 $ vertex sets $ W^{1,2, ..., q} $, $W^{1,2,...,q-1, q + 1} $ and so on up to $ W^{n-q + 1, ..., n-1, n} $. We also assign each vertex of these sets to unordered set of $q$ heroes of the DU problem according to $\binom{n}{q}$ possible combinations. In the same way as we did it for the first set, we enumerate vertices in the set $W^{\overbrace {i, j, k, l, ...}^\text {total $ q $ indices}} $ as $w_{\underbrace {1,2, ..., q}_\text {total $ q $ indices}}^{\overbrace{i, j, k, l, ...}^\text{total $ q $ indices}} $, $w_{ \underbrace {1,2, ..., q-1, q + 1}_\text{total $ q $ indices}}^{\overbrace{i, j, k, l, ...} ^\text{total $ q $ indices}} $,  ..., $w_{\underbrace {n-q + 1, ..., n-1, n}_\text{total $ q $ indices}}^{\overbrace { i, j, k, l, ...}^\text {total $ q $ indices}} $.\\
    We denote family of these sets as $ \mathcal{F}_q $. Thus, we have $ \binom{m}{q} $ sets with $ \binom{n}{q} $ vertices in each.
    We build additional edges between vertices $w_{\underbrace{a, b, c, ....}_\text{total $ q $ indices}}^{\overbrace{i, j, k , ...}^\text{total $ q $ indices}} $ and $w_{\underbrace{a',b', c', ...}_\text{total $ q $ indices}}^{\overbrace {i ', j', k ', ...}^\text{total $ q $ indices}}$ as follows. An edge is drawn if both of the following conditions are true:
    \begin{itemize}
        \item The vertices $w_{a, b, c, ...}^{i, j, k ...} $ and $ w_{a', b', c', ...}^{i', j', k', ...}$ correspond to different set of heroes ($ a \neq a '\lor b \neq b' \lor c \neq c', ... $)
        \item The vertices $ w_{a, b, c, ...}^ {i, j, k ...} $ and $ w_{a', b', c',...}^{i', j', k', ...} $ belong to different sets of the family $ F_q $ ($ i \neq i'\lor j \neq j' \lor k \neq k', ... $)
    \end {itemize}
    We also build all possible edges between every pair of vertices of the sets $ \mathcal{F} $ and $ \mathcal{F}_q $. We assign a weight $0$ to all those edges. Indeed, any maximal clique contains one vertex from each set of the families $\mathcal{F}$ and $\mathcal{F}_q$.
    %-------------------------------------------- остановился на процессе вычитки здесь
    We assign a big weight $N$ to edges $ (v_a^y, w_{a, b, c, ...}^ {i, j, k, ...}), (v_b^y, w_{a, b, c , ...}^{i, j, k, ...}),(v_c^y, w_{a, b, c , ...}^{i, j, k, ...}),... $, where subscipt of vertex $v$ is in the set of subcripts of vertex $w$.\\
    We are going to show that any maximal clique $\mathcal{C}$ of graph $G'$ has a set of vertices from the family $\mathcal{F}$ corresponding to a set of $m$ heroes, and at the same time $\mathcal{C}$ contains a set of vertices from the family $ \mathcal{F}_q $ corresponding to all combinations of $q$ heroes chosen from the team of $m$ heroes.
    In the clique $\mathcal{C}$, the edges connecting the vertices from the families $ \mathcal{F} $ and $ \mathcal{F}_q $ make a total contribution to the weight of $\mathcal{C}$, equal to $ q \binom{n}{q} N $.
    The clique includes exactly $ q \binom{n}{q} $ edges with additional big constant weight $N$: $q$ edges incident to each of $\binom{n}{q}$ vertices from the set $ \mathcal {F}_q $.
    %It is not hard to see that when a vertex in a clique belongs to the family $ \mathcal{F}_q $, and does not belong to a set of $q$ vertices from $ \mathcal {F} $, then the clique $\mathcal{C}$ contains at least one edge less with an additional big constant weight $ N $ among the edges of the clique.
    If theres is a vertex from $ \mathcal{F}_q $ which doesn't correspond to a subset of $q$ heroes of a chosen team of $m$ heroes, then the clique $\mathcal{C}$ contains at least one edge less with an additional big constant weight $ N $ among its edges.
    Therefore in this case the clique $\mathcal{C}$ doesn't have the maximum weight. Thus, the statement is proved. \\
    Note that adding weights on the edges that are small compared to $N$ preserves the truth of the statement. Since $N$ is chosen arbitrarily, we can assume that all values of heroes powers and bonuses are small compared to $N$.\\
	We add values of bonuses the edges between $\mathcal{F} $ and $ \mathcal{F}_q $. If the team has heroes corresponding to the subcripts of vertices from both $ \mathcal {F}$ and $ \mathcal {F}_q $, then this bonus will be included in the weight of the MEWC.
    Since all the cliques under consideration have the same number of edges with an additional weight $N$, the maximum clique will be the one where the sum of the heroes' powers (the sum of the weights of the edges between the vertices of the $ \mathcal{F} $ family) and bonuses (the weights of the edges between the vertices of the families $ \mathcal{F} $ and $ \mathcal {F}_q $ without taking into account the constants $ N $) is maximal.\\
Thus, the weight of the clique corresponds to the total value of the team, wherefrom the polynomial-time reduction is clear $\blacksquare$
\end{proof}
\section{Model application for real data}
\label{SectionComputationalResults}
We have applied the proposed model to analysis of real world DU problem. Note that our result should not be considered as an objective assessment of the quality of the team of heroes. The reason is the inevitable simplification of the heroes’ power as well as the influence that the alliances have. Each hero in Underlords has a certain ability, which is activated when various conditions are satisfied.
In addition, abilities have some recharge time. Alliance abilities and bonuses are also very diverse in their influence on the game -- they can cause damage, heal allies, prevent enemies from using their abilities, and other. Fortunately, the game has a system of five ``tiers", arranged so that the characters inside the tier are approximately equal by strength.\\
Within the simplified model, we accept the following:
\begin{enumerate}
\item The forces of all the heroes of the first tier are equal to 1,  the second -- 2, the third -- 3, the fourth -- 4, the fifth -- 5;
\item Alliances give the same percentage bonus to everyone they equally influence;
\item The alliance bonus is approximately 10-30 percent of the hero’s power.
\end{enumerate}

Information about heroes power and a structure of alliances is presented in the table \ref{table:aliances}.
A full table of alliance bonuses  $e_{ijk} $ can be found in our public \href{https://github.com/aponom84/UnderLords/blob/master/UnderLordsData.xlsx}{repository} \cite{UnderLordsInput}.

\begin{table}
\center
\resizebox{!}{9cm} {
\begin{tabular}{llrl}
{\#} &                 Heroes &  Power &                       Alliances \\
\midrule
0  &                 tusk &      1 &               savage, warrior  \\
1  &           venomancer &      1 &               scaled, summoner \\
2  &         shadow demon &      1 &               demon, heartless \\
3  &          drow ranger &      1 &    heartless, hunter, vigilant \\
4  &          bloodseeker &      1 &           blood-bound, deadeye \\
5  &         nyx assassin &      1 &               assassin, insect \\
6  &       crystal maiden &      1 &                    human, mage \\
7  &                 tiny &      1 &           primordial, warrior  \\
8  &             batrider &      1 &                  knight, troll \\
9  &               magnus &      1 &                  druid, savage \\
10 &             snapfire &      1 &                 brawny, dragon \\
11 &           arc warden &      1 &           primordial, summoner \\
12 &                razor &      1 &               mage, primordial \\
13 &               weaver &      1 &                 hunter, insect \\
14 &              warlock &      1 &  blood-bound, healer, warlock  \\
15 &               dazzle &      2 &                  healer, troll \\
16 &         earth spirit &      2 &               spirit, warrior  \\
17 &         storm spirit &      2 &                   mage, spirit \\
18 &         witch doctor &      2 &                troll, warlock  \\
19 &          bristleback &      2 &                 brawny, savage \\
20 &     legion commander &      2 &                champion, human \\
21 &        queen of pain &      2 &                assassin, demon \\
22 &     nature's prophet &      2 &                druid, summoner \\
23 &                 luna &      2 &               knight, vigilant \\
24 &           windranger &      2 &               hunter, vigilant \\
25 &            ogre magi &      2 &       blood-bound, brute, mage \\
26 &                pudge &      2 &            heartless, warrior  \\
27 &          beastmaster &      2 &                 brawny, hunter \\
28 &         chaos knight &      2 &                  demon, knight \\
29 &              slardar &      2 &               scaled, warrior  \\
30 &              abaddon &      3 &              heartless, knight \\
31 &                viper &      3 &               assassin, dragon \\
32 &           juggernaut &      3 &               brawny, warrior  \\
33 &         ember spirit &      3 &               assassin, spirit \\
34 &                   io &      3 &              druid, primordial \\
35 &         shadow fiend &      3 &                demon, warlock  \\
36 &                lycan &      3 &        human, savage, summoner \\
37 &          broodmother &      3 &               insect, warlock  \\
38 &            morphling &      3 &               mage, primordial \\
39 &          lifestealer &      3 &               brute, heartless \\
40 &           omniknight &      3 &          healer, human, knight \\
41 &          terrorblade &      3 &                  demon, hunter \\
42 &        shadow shaman &      3 &                summoner, troll \\
43 &               enigma &      3 &               primordial, void \\
44 &     treant protector &      3 &                   brute, druid \\
45 &                 doom &      4 &                   brute, demon \\
46 &            disraptor &      4 &               brawny, warlock  \\
47 &          void spirit &      4 &                   spirit, void \\
48 &               mirana &      4 &               hunter, vigilant \\
49 &           tidehunter &      4 &               scaled, warrior  \\
50 &            necrophos &      4 &    healer, heartless, warlock  \\
51 &           lone druid &      4 &        druid, savage, summoner \\
52 &                 sven &      4 &          human, knight, scaled \\
53 &                slark &      4 &               assassin, scaled \\
54 &     templar assassin &      4 &       assassin, vigilant, void \\
55 &  keeper of the light &      4 &                    human, mage \\
56 &                  axe &      5 &                  brawny, brute \\
57 &        faceless void &      5 &                 assassin, void \\
58 &            sand king &      5 &                 insect, savage \\
59 &                 lich &      5 &                heartless, mage \\
60 &               medusa &      5 &                 hunter, scaled \\
61 &        dragon knight &      5 &          dragon, human, knight \\
62 &        troll warlord &      5 &                troll, warrior  \\
\bottomrule
\end{tabular}
}
\caption{Heroes power and alliances  structure}
\label{table:aliances}
\end{table}

%%%%%%%%%%%%%%%%%%%%%%%%%%%%%%%%%%%%%%%%%%%%%%
%  RESULT TABLE
%%%%%%%%%%%%%%%%%%%%%%%%%%%%%%%%%%%%%%%%%%%%%%
We provide a solution of the linear integer programming model defined by the system of inequalities \eqref{eq:DUIP}, as a table \ref{table:solution}.

\begin{table}
\resizebox{16cm}{!} {
\begin{tabular}{l| *{8}{p{1.6cm}} | *{3}{ p{1cm}} }
{Hero} &                   &               &                &                &                &                  &                  &                  &  Alliance contribution &  Hero power &   Sum \\
\midrule
broodmother   &  heartless 2 +0.3  &  human 2 +0.3  &  insect 2 +0.3  &  scaled 2 +0.6  &   troll 2 +0.3  &  warlock  2 +0.6  &  warlock  4 +0.6  &                   &                  3.0 &           3 &   6.0 \\
disruptor     &  heartless 2 +0.4  &  human 2 +0.4  &  insect 2 +0.4  &  scaled 2 +0.8  &   troll 2 +0.4  &  warlock  2 +0.8  &  warlock  4 +0.8  &                   &                  4.0 &           4 &   8.0 \\
dragon knight &  heartless 2 +0.5  &  human 2 +0.5  &  insect 2 +0.5  &  knight 2 +1.0  &  scaled 2 +1.0  &     troll 2 +0.5  &  warlock  2 +0.5  &  warlock  4 +0.5  &                  5.0 &           5 &  10.0 \\
lich          &  heartless 2 +0.5  &  human 2 +0.5  &  insect 2 +0.5  &  scaled 2 +1.0  &   troll 2 +0.5  &  warlock  2 +0.5  &  warlock  4 +0.5  &                   &                  4.0 &           5 &   9.0 \\
medusa        &  heartless 2 +0.5  &  human 2 +0.5  &  insect 2 +0.5  &  scaled 2 +1.0  &   troll 2 +0.5  &  warlock  2 +0.5  &  warlock  4 +0.5  &                   &                  4.0 &           5 &   9.0 \\
necrophos     &  heartless 2 +0.4  &  human 2 +0.4  &  insect 2 +0.4  &  scaled 2 +0.8  &   troll 2 +0.4  &  warlock  2 +0.8  &  warlock  4 +0.8  &                   &                  4.0 &           4 &   8.0 \\
sand king     &  heartless 2 +0.5  &  human 2 +0.5  &  insect 2 +0.5  &  scaled 2 +1.0  &   troll 2 +0.5  &  warlock  2 +0.5  &  warlock  4 +0.5  &                   &                  4.0 &           5 &   9.0 \\
sven          &  heartless 2 +0.4  &  human 2 +0.4  &  insect 2 +0.4  &  knight 2 +0.8  &  scaled 2 +0.8  &     troll 2 +0.4  &  warlock  2 +0.4  &  warlock  4 +0.4  &                  4.0 &           4 &   8.0 \\
troll warlord &  heartless 2 +0.5  &  human 2 +0.5  &  insect 2 +0.5  &  scaled 2 +1.0  &   troll 2 +1.0  &  warlock  2 +0.5  &  warlock  4 +0.5  &                   &                  4.5 &           5 &   9.5 \\
witch doctor  &  heartless 2 +0.2  &  human 2 +0.2  &  insect 2 +0.2  &  scaled 2 +0.4  &   troll 2 +0.4  &  warlock  2 +0.4  &  warlock  4 +0.4  &                   &                  2.2 &           2 &   4.2 \\
\bottomrule
\end{tabular}
}
\caption{Optimal team structure for the Dota Underlords game with all active bonuses. An integer number following an alliance name means bonus level.}
\label{table:solution}
\end{table}
\section{Conclusion}
\label{SectionConclusion}
In this paper, we demonstrated that win in a video game can be based on using linear integer programming model.
We have described gameplay of the Dota Underlords, and have proposed a formal model as a set of inequalities \eqref{eq:DUIP} for one possible simplification of the real gameplay. We have proved that the proposed model is NP-hard, while decision version is NP-complete. In addition, we introduce a polynomial time reduction from DU problem to MEWC. Also we have made assumptions about heroes strengths, and values of alliance bonuses. Based on this data we have found the exact solution of the proposed model with using IP-solver. \\
The values of heroes  strengths, alliance bonuses, and  the results of solved the IP-model in the form of a Jupyter-notebook can be found in our pubic repository \cite{UnderLordsInput}.
\section{Acknowledgements}
The article was prepared within the framework of the Basic Research Program at the National Research University Higher School of Economics (HSE).

\bibliographystyle{unsrt}
\bibliography{references}

\end{document}

