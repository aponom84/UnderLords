\documentclass{article}
\usepackage{arxiv}
\usepackage[utf8]{inputenc} % allow utf-8 input
\usepackage[T1]{fontenc}    % use 8-bit T1 fonts
\usepackage{hyperref}       % hyperlinks
\usepackage{url}            % simple URL typesetting
\usepackage{booktabs}       % professional-quality tables
\usepackage{amsfonts}       % blackboard math symbols
\usepackage{nicefrac}       % compact symbols for 1/2, etc.
\usepackage{microtype}      % microtypography
\usepackage{lipsum}

\usepackage{amsmath, amsfonts, amsthm,amssymb}
\usepackage[english, russian]{babel}

\newtheorem{theorem}{Теорема}
\newtheorem{lemma}{Лемма}
\newtheorem{corollary}{Следствие}
\newtheorem{proposition}{Предложение}
\newtheorem{assertion}{Утверждение}

% \title{Theoretical Analysis of Random Geometric Hierarchical K-Graph}
 \title{NP-completeness and LP formulation of Dota Underlords problem}


\author{
  Alexander A. ~Ponomarenko, Dmitry S. ~Sirontkin\thanks{Use footnote for providing further
    information about author (webpage, alternative
    address)---\emph{not} for acknowledging funding agencies.} \\
  Laboratory of Algorithms and Technologies for Network Analysis\\
  National Research University Higher School of Economics \\
  Nizhny Novgorod, Russia\\
  \texttt{aponomarenko@hse.ru, dsirotkin@hse.ru} \\
  %% examples of more authors
  %% \AND
  %% Coauthor \\
  %% Affiliation \\
  %% Address \\
  %% \texttt{email} \\
  %% \And
  %% Coauthor \\
  %% Affiliation \\
  %% Address \\
  %% \texttt{email} \\
  %% \And
  %% Coauthor \\
  %% Affiliation \\
  %% Address \\
  %% \texttt{email} \\
}
%

\begin{document}
\maketitle

\begin{abstract}
Similarity searching has a vast range of applications in various fields of computer science. One of the promising algorithm for approximate nearest neighbor search is
Here we provide proofs of three main properties of random structure.
We provide analysis of the search complexity and estimate dependency of accuracy from input parameters. 
Вставить охуенную анотацию! Чтобы, все сука сразу обкончались!
\end{abstract}


% keywords can be removed
\keywords{Random geometric graphs \and Nearest Neighbour Search \and Metric Space \and Big Data \and Fuzzy Search \and Similarity Search \and Machine Learning \and Data Structures \and Theoretical Analysis}


\section{Описание Dota Underlords}
%Here will be our great introduction that will inspire everybody who reads it.

По ходу игры восемь игроков составляют команду из <<героев>> - существ, способных сражаться друг с другом на игровой карте. У каждого из героев есть базовые параметры - здоровье, урон, скорость атаки и прочие, а также особоая способность, которая определяет его роль в игре. Каждый герой принадлежит к двум или более <<альянсам>> - классам, в которые входят несколько героев. так, например, герой Enchantress принадлежит одновременно к альянсу <<друиды>> и к альянсу <<хищники>>. При наборе нескольких героев из одного альянса (для каждого альянса это число индивидуально) игрок получает бонус, который выражается в усилении всех героев из альянса, усилении всех своих героев или ослаблении всех героев соперника. Последнее может быть интерпретировано как относительное усиление своих героев и поэтому на протяжении работы будут рассматриваться только первые два случая. Следует отметить, что для одного альянса может быть несколько бонусов, которые открываются разным количеством героев соответствующего альянса, при этом они могут быть разного типа.

Также в ходе игры можно усилять своих героев до более высоких уровней или путём покупки внутриигровых предметов. В рамках данной работы эти аспекты учитываться не будут.

Таким образом, сила команды игрока определяется как:

\begin{enumerate}
    \item Силой выбранных героев
    \item Бонусами от альянсов в которых они состоят
\end{enumerate}

Как оказывается, данную задачу можно представить как задачу комбинаторной оптимизации. В рамках данной работы мы показываем её NP-полноту, приводим её формулировку как задачу ЛП и решаем её симплекс-методом для частного случая Dota Underlords.

\section{Перевод задачи в язык LP}

\subsection{Простейшая постановка задачи}

Формализуем задачу следующим образом:

Будем считать, что всего у нас в есть $p$ героев на выбор. Будем считать, что сила некоторого $i$-го героя опредяляется за некоторую неотрицательную величину $s_i$. За $x_i$ будем обозначать принадлежность героя выбранной команде --- $x_i = 1$, если $i$-й герой принадлежит набранной команде и $x_i=0$ в противном случае. Тогда условие того, что в команде не более, чем $n$ героев можно записать в виде $\sum_{i=0}^p x_i \leq n$. Тогда в простейшей форме данную задачу можно сформулировать следующим образом:

\begin{equation}
\begin{gathered}
    max \sum_{i=0}^p x_i s_i \\
    \sum_{i=0}^p x_i \leq n \\
    p, s_i, n - const \\
    x_i - \{0, 1\}
\end{gathered}
\end{equation}

В данной постановке задача решается элементарно - достаточно взять $n$ элементов с наибольшими весами

\subsection{Постановка задачи с альянсами}

В рамках нашей модели мы рассмотрим два типа альянсов - те которые дают бонусы своим членам и те, которые дают бонусы всем героям игрока. Пусть всего $r$ альянсов. Мы будем считать, что бонус от альянса увеличивает силу героя на некоторую величину $b_{ij}$, где $i$ - номер героя, а $j$ - номер альянса. При этом, альянсы рассмотренных видов отличются только тем, что в альянсах, дающих бонус своим членам, величина $b_{ij}$ равна нулю тогда и только тогда, когда $i$-й герой не принадлежит $j$-му альянсу.

Таким образом, необходимо, чтобы модель учитывала, когда бонус от альянса активирован, а когда нет. Введём индикаторную величину $I_j$, которая равна единице, если бонус от $j$-го альянса активирован и нулю в противном случае. Условие того, что все элементы альянса взяты, и бонус активирован можно записать как $I_j - x_i \leq 0$ при $i$-м элементе лежащем в $j$-m альянсе. Для удобства мы вводим величину $e_{ij}$, которая равна нулю, если $i$-й элемент не лежит в $j$-м альянсе и единице в противном случае. Тогда данное условие можно переписать как $I_j - x_i - e_{ij} + 1 \leq 0$

Величина $I_j$ может быть равна единице тогда и только тогда, когда все соответствующие ей $x_i$ и $e_{ij}$ равны единице. Поскольку в максимизирующую функцию это величина входит с положительным коэффицентом, то она в этом случае в точности будет равна единице.

Таким образом, после введения в модель альянсов, она выглядит следующим образом.

\begin{equation}
\begin{gathered}
    max \sum_{i=0}^p x_i s_i + \sum_{j=0}^r \sum_{i=0}^p I_{j} b_{ij} \\
    \sum_{i=0}^p x_i = n \\
    \forall j : I_j - x_i - e_{ij} + 1 \leq 0 \\
    p, r, n, b_{ij}, e_{ij} - const \\
    x_i, I_j - \{0, 1\}
\end{gathered}
\end{equation}

\subsection{Доказательство NP-полноты задачи Dota Underlords с альянсами}

Покажем, что данная задача в указанной в предыдущем разделе формулировке является NP-полной. Рассмотрим её частный случай, удовлетворяющий следующему ограничениям 

\begin{enumerate}
    \item Веса элементов равны нулю
    \item Никакой элемент не принадлежит более чем одному классу
    \item Каждый элемент принадлежит хотя бы одному класу
\end{enumerate}

Очевидно, в рамках данной задачи брать только часть элементов из класса смысла не имеет, поскольку это не увеличит сумарный вес. Таким образом, мы либо берём весь класс, либо не берём. Это можно проинтерпретировать как задачу о рюкзаке, где <<предметом>>, который можно положить в рюкзак, является класс, его вес --- это количество элементов в нём, <<стоимость>> --- суммарный бонус класса, а <<вместимость рюкзака>> --- общее количество элементов, которые необходимо взять.

Данная задача является классической NP-полной задачей и сводится к задаче <<Dota Underlords>>.

\subsection{Постановка задачи с частичными бонусами от альянсов}

В <<Dota Underlords>> для активации бонуса иногда можно собрать не весь альянс, а некоторую его часть - треть или половину. Более, того, могут существовать несколько подобных бонусов, например, один бонус за собранную треть альянса и бонус за собранные две трети альянса. Другим подобным случаем является случай, когда в игре присутствует больше героев в некотором альянсе, чем требуется для активации бонуса. 

% Пусть для активации некоторого $k$-го бонуса $j$-го альянса требуется собрать $f_{jk}$ элементов этого альянса. Тогда можно пронумеровать все возможные бонусы от первого до $t_j$-го внутри каждого альянса. Введём индикаторную величину $I_{jk}$, которая показывает, что k-й бонус j-го альянса активирован. Тогда величина $e_{ijk}$ показывает бонус для i-го элемента, если активирован k-й бонус j-го альянса. Учитывая, что постановка задачи с <<чистыми>> альянсами является частным случаем данной постановки задачи, достаточно записать максимизирующую функцию как $\sum_{i=0}^p x_i s_i + \sum_{j=0}^t \sum_{i=0}^p \sum_{k=0}^{t_{j}} e_{ijk} I_{jk}$. Запишем неравенство, запрещающее индикаторной величине принимать значение 1, если менее $f_{jk}$ героев из альянса выбрано в команду: $\sum_{i=0}^n e_{ij} x_{i} - f_{jk} + I_{jk} - 1 \leq 0$
Данную ситуацию предлагается моделировать с помощью введения 3-х индексного тензора $e_{ijk} \in \mathbb{R}$ означающего бонус за присутствие в команде героя $i$, входящего в альянс $j$, и данный бонус активируется, когда в команде присутствуют не меннее $k$ героев из того же альянса $j$, другими словами, $e_{ijk}$ это $k$-й бонус альянса $j$ для героя $i$. Данный тензор будет сильно разрежан, поскольку как правило, один герой пренадлежит не больше чем 3-м альянсами.  Контролировать вхождения бонуса $e_{ijk}$ в общую силу команды предлагается с помощью управляющей бинарной переменной $I_{ijk}$.
Так мы можем записать целевую функцию как следующую сумму $\sum_{i=1}^{n} x_i s_i + \sum_{i=1}^{n} \sum_{j=1}^{t} $.
Связь переменных $x_{i}$ и $I_{ijk}$ задаётся неравенствами
$\forall{i,j,k} :  \sum_{i'=1}^{n} a_{i'j} x_{i'} - k \ge M( I_{ijk}  - 1)$. \textbf{Здесь надо вставить пару предложений про эти неравенства, в чём логика этой связи}
Они недают бинарной переменной $I_{ijk}$ принимать значение 1, если в решение входит меньше чем $k$ героев входящих в альянс $j$. Когда решение содержит героев из альянса $j$ меньше чем $k$, левая часть этого неравенства отрицательная, поэтому чтобы неравентсва соблюдались правая часть должна быть ещё меньше. Такое возможно только, когда бинарная $I_{ijk}$ равно нулю. В этом случае правая часть равна $-М$, где $М$ константа заведомо большая, чем $k$, то есть больше, чем максимальный размер альянса $q$.
Разумно требовать, чтобы бонус для героя $i$ мог быть активирован ($I_{ijk} = 1$ ), только когда герой $i$ входит в решение. Это задаётся неравенствами $\forall{i,j,k} :  I_{ijk}  \le x_i$. Также мы хотим, чтобы бонус $e_{ijk}$ был активирован, только для если герой $i$ принадлежит альянсу $j$. Для этого мы в модель включили неравенства $\forall{i,j,k} :  I_{ijk}  \le a_{ij}$.   

 
Общая система уравнений выглядит следующим образом:
\begin{equation}
\begin{gathered}
\textbf{Целевая функция:}\\
max \sum_{i=1}^{n} x_i s_i + \sum_{i=1}^{n} \sum_{j=1}^{t}  \sum_{k=1}^{q} e_{ijk} I_{ijk} \\

\textbf{Ограничения на входные данные}\\
    
\forall{j} : \sum_{i=1}^n a_{ij} \le q \\

Ограничения на управляющие переменные \\
\forall{i,j,k} :  \sum_{i'=1}^{n} a_{i'j} x_{i'} - k \ge M( I_{ijk}  - 1) \\
\sum_{i=1}^n x_i \le m   \\ 

% \sum_{i=0}^{n-1} a_{ij} x_{i} < k \\ 

\forall{i,j,k} :  I_{ijk}  \le x_i \\

\forall{i,j,k} :  I_{ijk}  \le a_{ij} \\

\text{Управляющие переменные:} \\
I_{ijk} \in \{0, 1\} \text {, 1 – если для героя } i \text{, активирован} k\text{-й бонус } j \text{-го альянса,} \\
x_i  \in \{0, 1\} \text{, 1 -- если герой } i \text{– входит в решение} \\

\textbf{Константы:} \\
n \in \mathbb{N} \text{ -- число героев,} \\
m \in \mathbb{N} \text{ -- максимальный размер команды}\\
q \in \mathbb{N} \text{ -- максимальный размер одного альянса,} \\

s_i  \in \mathbb{R} \text{–- сила героя } i, \\
e_{ijk} \in \mathbb{R} \text{ -- бонус для героя } i \text{,  если активирован } k
\text{-й бонус } j \text{-го альянса} \\
a_{ij} \in \{0, 1\} \text{, 1 -- если герой } i \text{ входит в альянс } j \\ 
\end{gathered}
\end{equation}

\subsection{Практическое применение для реальной задачи Dota Underlords}
  
Мы применяем данную модель для анализа реальной задачи Dota Underlords. Отметим, что полученные результаты не стоит считать некоторой объективной оценкой качества команды героев. Причина состоит в неизбежном упрощении сил героев и влияния, которое оказывают альянсы. Каждый герой в Underlords обладает некоторой способностью, которая активируется при заполнении шкалы маны и обладает некоторым временем перезарядки. Способности и бонусы альянсов также весьма разнообразны по своему влиянию на игру - они могут наносить урон, лечить союзников, мешать врагам пользоваться своими способностями и прочее. К счастью, в игре есть система из пяти <<ярусов>>, устроенная так, что герои внутри яруса примерно равны по силе.

В рамках упрощённой модели мы принимаем следующее.

1) Силы всех героев первого яруса равны 1, второго 1.5, третьего - 2, чевертого - 2.5, пятого - 3

2) Все альянсы дают один и тот же мультипликативный бонус 1.1. Если у альянса больше одного уровня влияния на героев (например альянс воинов даёт своим героям последовательно +10, +15 и + 25 к броне), то второй уровень даёт мультипликативный бонус 1.2, третий - 1.3.

Особым случаем является альянс Scrappy, которой даёт бонус на своём первом уровне одному своему случайно выбранному члену. Мы считаем этот бонус равномерно распределённым между всеми членами альянса в рамках общей модели и считаем как и прочие бонусы первого уровня. Бонус второго уровня мы в этом альянсе считаем так же, как в других.

\section{Частные разрешимые случаи}

Как и многие подобные задачи, задача Dota Underlords становится полиномиально разрешимой при определённых ограничениях. 

Рассмотрим простейший нетривиальный случай

\begin{enumerate}
    \item Каждый герой принадлежит ровно двум альянсам
    \item Каждый альянс содережит ровно двух героев
    \item Сила каждого героя равна $a$
    \item Бонус от каждого альянса на каждого героя одинаков и равен $\frac{b}{2}$
\end{enumerate}

Таким образом, суммарный бонус от сбора альянса в данной модели равен $b$. 

\begin{theorem}
    Задача Dota Underlords в указанной формулировке решается за $O(nb)$, если $b$ - натуральное и NP-полна, если числа $a$ и $b$ - несравнимы. 
\end{theorem}

\begin{proof}
 Данную задачу удобно перефомулировать на языке теории графов. Обозначим героев за вершины графа, а отношение "находятся в одном альянсе" - за ребро. Тогда в данном графе степени всех вершин равны 2 и он разбивается на несколько (возможно один) циклов, несвязных между собой. Т.к. число элементов в искомом наборе фиксировано, то итоговая целевая сумма зависит от числа рёбер, порождаемых в данном графе множеством взятых вершин-героев. Таким образом, если мы берём из некоторого цикла множество вершин мощности $k$, которое не совпадает с множеством всех вершин цикла, то мы можем получить самое большее бонус $(k-1)b$ - взяв вершины, порождающие путь в данном цикле. В случае, если мы берём весь цикл, то мы получаем бонус $kb$. таким образом, для достижения наибольшего бонуса необходимо, чтобы все взятые вершины порождали в графе некоторое множество циклов (возяожно ноль) и не более одного пути. При этом, случай, когда пути вообще нет лучше случая, когда он есть. В случае, если нам надо взять всего $n$ элементов и мы смогли взять $n$ вершин, порождающих исключительно циклы, мы получаем итоговую сумму $n(a+b)$. Если же присутствует путь, то мы получаем сумму $na+(n-1)b$. Отметим, что фактически необходимо проверить возможность построения набора, порождающих исключительно циклы, т.к. построить набор, порождающий циклы и путь --- элементарно, достаточно добавлять циклы в набор в произвольнос порядке, пока не окажется, что мы хотим добавить цикл размера большего, чем оставшееся количество вершин. Тогда вместо этого добавляем произвольный путь из этого цикла. 
 
 Данная задача легко сводится к задаче о рюкзаке с целыми весами и решается методом динамического программирования. Действительно, размеры циклов в этой задаче соответсвуют весам и объёму элементов, общее их количество в итоговом решении - объёму рюкзака. Таким образом, наобходимо лишь проверить, что в решении данного экземпляра задачи о рюкзаке рюкзак заполнится полностью.
 
 Сложность алгоритма  в данной задаче равна $O(nb)$ при целом $b$. Задача тогда решается методом динамического программирования.
 
 В случае произвольного $b$ данная задача представлет собой специфический случай задачи о сумме. Задачей о сумме называется задача в которой требуется определить, есть ли в данном множестве чисел подмножество с некоторой суммой $s$ (в каноническом случае - с суммой 0). Известно, что он - NP-полна Таким образом, задача Dota Underlords сводится к ней и значит NP-сложна. 
\end{proof}

\section{NP-completeness of the problem}


 


%%% Comment out this section when you \bibliography{references} is enabled.
\begin{thebibliography}{1}



\end{thebibliography}


\end{document}
